\documentclass{suturo}
\usepackage{graphicx}

\makeatletter
\newcommand{\chapterauthor}[1]{%
  {\parindent0pt\vspace*{-27pt}%
  \linespread{0}\small\begin{flushright}von: #1\end{flushright}%
  \par\nobreak\vspace*{0pt}}
  \@afterheading%
}
\makeatother

\begin{document}
\begin{titlepage}
	\centering
	{\scshape\LARGE Sudo tidy up my room \glqq Suturo\grqq{} \par}
	\vspace{1.5cm}
	{\scshape\Large Meilenstein 3\par}
	\vspace{1.5cm}
	{\huge\bfseries Projektbericht \par}
	\vspace{2.5cm}
	{\normalsize\bfseries Verfasst von \par}
	{\small\itshape Roman Haak, Maximilliam Bertram\par}
	{\small\itshape Alexander Link, Tammo W\"ubbena\par}
	{\small\itshape Vanessa Hassouna, Kevin St\"ormer, Hauke Tietjen\par}	
	{\small\itshape Alexander Haar \&  Max-Phillip Bahr\par}	
	\vspace{2.5cm}
	{\normalsize\bfseries Tutoren: \par}
	{\small\itshape Georg Bartels, Ferenc Balint-Benczedi \\ Daniel Be{\ss}ler \& Gayane Kazhoyan\par}


	\vfill
\end{titlepage}

\tableofcontents

\newpage

\section{Zielsetzung}
Womit hast du das Arichtekturbild gemacht? - Mit Paint!


\section{Ein tolles Kapitel}

\section{Noch ein tolles Kapitel}

\section{Das letzte geile Kapitel}

\end{document}