\documentclass{suturo}

\begin{document}
    \maketitle{Knowledge}{13.05.2017}{}{2}{}{}{}{}

\makeatletter
\newcommand{\chapterauthor}[1]{%
  {\parindent0pt\vspace*{-27pt}%
  \linespread{0}\small\begin{flushright}von: #1\end{flushright}%
  \par\nobreak\vspace*{0pt}}
  \@afterheading%
}
\makeatother

\section*{Zielsetzung}
\chapterauthor{Max-Phillip Bahr}
Das Ziel von Knowledge im zweiten Meilenstein ist das Klassifizieren und abspeichern von erkannten Gegenständen im Beliefstate sowie das bestimmen einer geeigneten GraspPose für jeden Gegenstand.

\section*{Probleme}
\chapterauthor{Max-Phillip Bahr}
\subsection*{Bestimmen der GraspPose vom Objekt}
Jedes Objekt muss mindestens eine GraspPose besitzen, an der es vom Roboter gegriffen werden kann.\\
\textbf{Lösung}: In unserer OWL-Ontologie haben wir für jedes Objekt manuell eine GraspPose eingespeichert. Diese wird dann von einer Node ausgelesen und weitergegeben.

\end{document}
